\documentclass{letter}

\signature{Savita Karthikeyan}

\address{Big Data Institute\\University of Oxford\\UK}
\begin{document}

\begin{letter}{Bioinformatics}

    \opening{Dear Editors,}

    I write on behalf of my co-authors to submit our original research work
    entitled
    \emph{'tsbrowse: an interactive browser for Ancestral Recombination
        Graphs'} to
    Bioinformatics,
    and request that it be considered for publication as an Application Note.

    The manuscript describes an open-source Python web-app to visualise
    fundamental
    entities
    of Ancestral Recombination Graphs (ARGs). ARGs are a topic of intense
    research
    interest in the field of genomics, following recent technological
    advancements
    which have made their inference from large-scale empirical datasets
    possible: see Grundler et al. (Science, 2025), Speidel et al. (Nature,
    2025)
    for recent ARG-based findings, and Nielsen et al. (Nature Reviews Genetics,
    2024) for a recent review of ARG inference.
    
    Visual inspection is a crucial part of data analysis pipelines. For omics,
    this
    vital
    infrastructure is provided by state-of-art genome browsers. The lack of
    tools
    for the
    visual inspection of ARGs is a major gap in a rapidly advancing field. With
    tsbrowse,
    we bridge this gap by presenting interactive views of the nodes,
    edges
    and mutations
    in an ARG. The manuscript describes the applications and architecture of
    tsbrowse, and
    demonstrates user-interactivity and scalability to millions of samples.

    We believe this work directly aligns with the scope of Bioinformatics, and
    that
    its
    readers will find tsbrowse of particular interest. As we show in the
    manuscript, visualising
    the information structure in ARGs greatly helps improve inference
    pipelines,
    making tsbrowse
    particularly useful for modern, large-scale genomic datatsets. We hope
    that, if
    published
    tsbrowse will drive further advancements in the areas of ARG visualisation
    and
    validation,
    and make these steps a routine part of ARG inference.

    \closing{Sincerely,}

\end{letter}
\end{document}