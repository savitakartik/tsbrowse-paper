\documentclass{letter}
\usepackage{xcolor}
\usepackage{hyperref}
\signature{Savita Karthikeyan}

\address{Big Data Institute\\University of Oxford\\UK}
\begin{document}

\begin{letter}{}

\opening{Dear Alison,}

We are delighted that the reviewers have found our manuscript suitable for publication 
in Bionformatics. We thank you, the editors and all the reviewers for the careful 
consideration of our manuscript, and value the kind words and constructive feedback 
provided on our work. We have now revised the manuscript based on the reviewers' 
suggestions. Below is our point-by-point response to the 
reviewers' comments in blue. We have incorporated as many suggestions as possible, 
and where unable to do so we have indicated the reason.\\

\hrule

Reviewer: 1

Comments to the Author \\

Summary: In this manuscript, the authors present tsbrowse, an open-source Python-based 
application for the interactive visualization of the components of Ancestral Recombination 
Graphs (ARGs). Because ARGs (in principle) contain all knowable information concerning 
coalescence and recombination in the history of a set of samples, there is great excitement 
in developing methods for inferring ARGs from empirical genomic sequence data, as well as 
for learning about evolutionary processes from patterns in inferred ARGs. A crucial component 
of both these endeavors is visualizing ARGs, which is difficult or unwieldy. Tsbrowse therefore 
represents an exciting contribution to the field which is likely to be of high utility and 
to be highly cited.

Opinion: This paper was straightforward, clear, and well written. I think it does a good job 
setting up the intellectual context for ARG-based research and advocating for the utility of 
tsbrowse. I had two minor, nitpicky comments, but overall, it was great.  Thanks for a fun and 
easy review!

1. I think the order in which the supplemental figures are cited doesn’t match their numbering 
(i.e., S4 is cited before S1)\\
\textcolor{blue}{Thanks to Reviewer 1 for pointing out this error. We have now rearranged the 
supplementary figures so that the citations appear in ascending order.}

2. In the discussion, the authors mention that methods for visualization of tree topology can 
typically only handle a few hundred nodes, but it might be worth doing a “(but see…)” for some 
of the approaches folks have taken for visualizing much larger trees (e.g., nextstrain, onezoom).\\
\textcolor{blue}{We agree with the reviewer-  it is important to acknowledge previous/on-going visualisation efforts 
for larger trees. We had already included a citation to the OneZoom project (and the Treenome browser) in the 
sentence that followed the one in question "… Visualisation of large-scale tree topologies with millions of nodes 
requires much more sophisticated approaches to capture topological features at different scales, 
and is an active research area". We are thankful to the reviewer for bringing Nextstrain to our 
attention, and have now included Hadfield et al. as a citation.}

Reviewer: 2\\

Comments to the Author\\

I have only a few minor comments:\\
1. "and storing as a compressed .tsbrowse file. (The .tsbrowse file is also a valid input for
tszip, a general utility for compressing tskit files.)" Sorry, but I don't understand this sentence. 
More precisely, both the compressed and the uncompressed files have the .tbrowse extension? If yes, 
that's a weird choice and I would like to see some justification. You should also make clear if tszip 
is a part of tskit or of twbrowse.\\
\textcolor{blue}{To clarify, the tszip is part of the tskit ecosystem. It is not part of the tsbrowse 
package but can be used to compress the tables output from the tsbrowse preprocess step. Thanks to 
the reviewer for this pointing out this potentially unclear statement. We have now deleted the text 
in brackets, which we believe isn't crucial for understanding the data model.}

2. "web-browser" - web browser\\
\textcolor{blue}{Fixed, thanks for pointing out the typographical error.}

3. "Fig S4, for example, shows a screenshot of tsbrowse". This is a matter of personal taste, but I do
not like to refer to a Figure in the supplementary materia just as if it were in the article; if the 
Figure is relevant, it should be in the article. If it's not relevant, that you should refer to it without
hinting that it is important.\\
\textcolor{blue}{We agree with the reviewer's opinion, and believe that Figure S4 is crucial 
in order to demonstrate the application of tsbrowse to large-scale datasets (which is a key design 
feature of the app). However, we are unable to move this figure to the main matter considering journal 
guidelines for page limit.}

4. Looking at the github repo, I can find the installation instruction using pip only. In my experience 
pip is quite picky. I strongly suggest you to provide tsbrowse via conda-forge/bio-conda or a containerized 
solution (e.g. docker).\\
\textcolor{blue}{Thanks to the reviewer for the suggestion to provide alternate installation solutions. 
We have now created a Docker container in order to ensure a smooth experience across computing environments. 
Links have been updated in the manuscript and app documentation}
\textcolor{red}{[TO DO; after release of new package]}

Reviewer: 3

Comments to the Author
This paper offers a browser-based visualisation tool for ancestral recombination graphs (ARGs), which is 
a data structure representing gene genealogies for a sample of DNA sequences. The complication in the p
resence of recombination is that these structures are not simple trees, but - because of the possibility 
of recombination - graphs with reticulated nodes, which makes exploratory data analysis difficult.

I agree with the authors that there is currently a lack of visualisation tools for ARGs, and I expect the 
current paper to provide a useful addition to the catalog of tools for ARG simulation and inference. Tsbrowse 
integrates with the widely used tskit ecosystem of software tools, developed by some of the current authors, 
which will facilitate a rapid uptake. The authors make a compelling argument for the need for visualisation 
tools - Fig S1 in particular shows that four different and popular tools for ARG inference are giving 
qualitatively completely different insights into their reconstructed ARGs (incidentally, it's a shame that 
that figure couldn't make it into the main paper, though I recognise that this journal has strict page limits).\\
\textcolor{blue}{We appreciate the reviewer's feedback on the utility of tsbrowse, and acknowledge the 
importance of S1 in highlighting the need for visualisation. However, considering the necessity of the 
existing main display items to establish the fundamental aspects of tsbrowse, we have decided to retain S1 
in the supplementary. We agree that S1 would have been a compelling main display item had the page/figure 
limits of the journal allowed for it.}

My main comment is that tsbrowse does not actually seem to be an ARG visualisation tool. It is a useful way 
to visualise mutations, tree-sequence edges, and summary statistics, but nowhere will you actually see a tree 
(much less an ARG). On rereading the paper I see that the manuscript has been carefully worded not to include 
such a claim, but a casual reader might expect tsbrowse to be able to do so, as other tools such as IcyTree can. 
I do get fed up with reviewers tossing additional requests on top of an already polished piece of work, but 
in this case I think it would be a helpful addition! - would it really be too much work for tsbrowse to be 
able to draw at least the local trees? (For researchers with smaller datasets, being able to interrogate the 
fine details of an ARG would be equally valuable as the complementary selling-points around scalability.) At 
the very least I would like to see the manuscript at least be a little more up-front about this omission.\\

\textcolor{blue}{We thank the reviewer for making this important observation, and agree that drawing 
(parts of) local trees would be a great feature. For smaller datasets, tools such as arg-visualiser 
(\url{https://pypi.org/project/tskit-arg-visualizer/}) and existing tskit infrastructure, for example 
the \texttt{draw\_text()} and \texttt{draw\_svg()} already provide this functionality. A key 
motivation for this work is to guide large-scale inference, and implementing tree visualisation that is interpretable 
at this scale would require significant thought towards design as well as considerable effort. We therefore 
think that implementing this feature is not feasible for this work and is better scoped as future work or as 
a separate endeavour. In the text, we do state that tsbrowse is a visualiser for the information structure 
in ARGs, which we hope implies that it is a visualiser for the properties of the ARG, rather than the ARG 
topology itself. We have also already explicitly stated in the Discussion that drawing local trees is an 
important future direction.}\\

Minor comments:\\

Manuscript p1: 'it is only with recent breakthroughs in inference methods...that practical application 
has become possible'. I appreciate the need to emphasise the recent increase in interest in inference for ARGs, 
but I think this is selling older methods a little short. There are ARG inference methods going back at least 
as far as Griffiths and Marjoram (1996) 'Ancestral Inference from Samples of DNA Sequences' JCB 3(4):479-502. 
True they are not scalable in the manner of recent methods, but they *are* practical applications, and they have 
the advantage that they actually sample from the claimed model without introducing any model heuristics.
\textcolor{blue}{Thanks to the reviewer for their feedback on this statement on early methods. We agree, 
and have revised the text to acknowledge the contributions of foundational work in ARG inference.}

Manuscript p1: Typo 'developements'.\\
\textcolor{blue}{Thanks, this is now fixed.}

Manuscript p2: Where you mention 'Bokeh', can you clarify what this is (e.g. with a citation)?\\
\textcolor{blue}{Yes, Bokeh is a Python library for interactive visualisation and a citation to it 
has been added.}
\textcolor{red}{TO FIX! Citation appears as a question mark!!.}

Supplement p1, Sec 1.4: "The ARG shown in Fig 4" should say "The ARG shown in Suppl. Fig 4" ?\\
\textcolor{blue}{Thanks, this is now fixed.}\\
\textcolor{red}{(TO DO: make sure to replace links in suppl pdf!)}

Fig S2 caption - 'Consortium (2015)' is citing incorrectly.\\
\textcolor{blue}{Fixed (removed this citation as it was not required)}

From the online documentation, Node page: \url{https://tskit.dev/tsbrowse/docs/stable/nodes.html}
"The histograms at the bottom show the distributions of node spans over different dimensions. 
The leftmost histogram summarises the span of nodes on the sequence; the middle plot summarises 
the span of nodes over time and the rightmost plot summarises the edge “area” defined as the product 
of sequence span and time span for each node."

I didn't understand this. The 'nodes' section of the visualiser only seems to provide one histogram, 
not three.\\

\textcolor{blue}{Thanks to the reviewer for pointing out this documentation error. The Nodes section 
does indeed have one histogram and the documentation has been updated to reflect this.}

Installation notes:

I had a few problems getting tsbrowse to work. There are two different sets of installation instructions:

1. https://pypi.org/project/tsbrowse/
2. https://tskit.dev/tsbrowse/docs/stable/contributing.html

I use a Mac+Homebrew+Chrome and Option 2 seems more fitting. ("python -m pip install tsbrowse" as in 
option 1 will fail because Homebrew's Python environment is externally managed.)

\textcolor{blue}{We apologise for any confusion on installation instructions. The recommended way to install 
tsbrowse is via pip within a virtual environment. We have fixed the documentation to show only one set of 
instructions (i.e, for the user) at the top.}

Bug(?):

I frequently had problems rendering the window in the 'Mutations' tab, which tended to get stuck at the 
circling wheel. When clicking the 'Mutations' button, my terminal returned the following error:

2025-05-27 12:29:41,646 [5441] ERROR    tornado.application: Exception in callback functools.partial(>, .wrapped() 
done, defined at \\
/Users/reviewer/myenv/lib/python3.13/site-packages/panel/io/server.py:153> 
exception=AbbreviatedException(, ValueError('Supplied cmap winter not found among matplotlib, bokeh, or colorcet 
colormaps.'), )>)

\textcolor{blue}{Thanks very much to the reviewer for bringing this bug to our attention. It appears to be a 
packaging error. An issue has been raised on the tsbrowse code repository 
\url{https://github.com/tskit-dev/tsbrowse/issues/248}. We are working on a solution, and will release a new 
Python package as soon as this is fixed.}\\


\hrule


We have also addressed the editorial comments and ensured the following:
\begin{itemize}
    \item \textcolor{red}{TO DO.} Code has been published on Zenodo and archival DOI updated in the code availability in addition 
        to code repository
    \item \textcolor{red}{TO DO.} Manuscript adheres to mandatory template format: \textcolor{red}{INSERT TEMPLATE LINK}
    \item \textcolor{red}{TO DO.} Supplementary material attached as separate pdf to main matter
    \item \textcolor{red}{TO DO.} Following .tex files included: .tex, .cls, .bib, .bst and .ps 
    \item \textcolor{red}{TO DO.} Mark up of changes in the revised text included as \textcolor{red}{INSERT NAME OF DOC}
    \item \textcolor{red}{TO DO.} Conflicts of interests declared for all authors
\end{itemize}

We look forward to hearing from you about our revised manuscript. Please let us know if you require any 
further clarification.

On behalf of all tsbrowse authors,
\closing{Sincerely,}

\end{letter}
\end{document}